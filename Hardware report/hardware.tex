\documentclass{article}

\usepackage[margin=1in]{geometry} % Set margins to 1 inch
\usepackage{graphicx} % Allows including images
\usepackage{float} % Allows for precise placement of figures
\usepackage{amsmath} % Allows for math equations
\usepackage{siunitx} % Allows for SI units
\usepackage{placeins} % Makes sure images are in their respective sections by \FloatBarrier

\begin{document}

\title{Hardware Assignment Documentation\\ \large{Rayi Giri Varshini\\EE22BTECH11215}}
\author{}
\date{}
\maketitle
\maketitle

\section*{Aim}
	Random Numbers Generation using shift registers.
\section*{Description}

	7-segment displays provide a convenient way of displaying numerical information from zero to nine as they basically consist of a load of light emitting diodes connected together within a single indicator package. The circuit uses 5V from Micro USB and this is Vcc for the circuit. The X-OR gate - a digital logic gate that gives a true output when the number of true inputs is odd, Flip flops and Decoder are to be connected to the clock - IC 555 to generate random numbers in the 7-segment displayer. The numbers follow a sequence of - 1, 3, 7, 14, 15, 13, 10, 5, 11, 6, 12, 9, 2, 4 ,8, 1, 3... This runs till the source is connected. Direct connection from 7-segment displayer to the source voltage should not be done as the display is very sensitive a resistor of 1k$\Omega$ should be added such that no short circuit occurs. The clock has a certain period depending on the values of capacitors and resistors. The random generator generates numbers from 1 to 15 random fashioned.  
    
\section*{Components Used}
\begin{enumerate}
    \item Breadboard
    \item Seven Segment Display - Common Anode
    \item 7447 Decoder
    \item 7474 Flip Flops
    \item 7486 X-OR gate
    \item 555 IC
    \item Resistor 10M$\Omega$
    \item Resistor 1K$\Omega$
    \item Capacitor 100nF
    \item Capacitor 100nF
    \item Jumper wires
    \item Micro USB
\end{enumerate}

\begin{figure}[ht]
	\centering
	\includegraphics[width=0.7\linewidth]{images/image1.jpg}
	\caption{Image of circuit}
	\label{fig:view}
\end{figure}
\FloatBarrier

\begin{figure}[ht]
	\centering
	\includegraphics[width=0.7\linewidth]{images/image2.jpg}
	\caption{Image of circuit}
	\label{fig:view}
\end{figure}
\FloatBarrier

\section*{Block Diagram}
\begin{figure}[ht]
	\centering
	\includegraphics[width=0.7\linewidth]{images/blockdiagram.png}
	\caption{Block Diagram for circuit connections}
	\label{fig:view}
\end{figure}
\FloatBarrier
\end{document}
